\let\negmedspace\undefined
\let\negthickspace\undefined
\documentclass[journal,12pt,twocolumn]{IEEEtran}
\usepackage{cite}
\usepackage{amsmath,amssymb,amsfonts,amsthm}
\usepackage{algorithmic}
\usepackage{graphicx}
\usepackage{textcomp}
\usepackage{xcolor}
\usepackage{txfonts}
\usepackage{listings}
\usepackage{enumitem}
\usepackage{mathtools}
\usepackage{gensymb}
\usepackage{comment}
\usepackage[breaklinks=true]{hyperref}
\usepackage{tkz-euclide} 
\usepackage{listings}
\usepackage{gvv}                                        
\def\inputGnumericTable{}                                 
\usepackage[latin1]{inputenc}                                
\usepackage{color}                                            
\usepackage{array}                                            
\usepackage{longtable}                                       
\usepackage{calc}                                             
\usepackage{multirow}                                         
\usepackage{hhline}                                           
\usepackage{ifthen}                                           
\usepackage{lscape}
\newtheorem{theorem}{Theorem}[section]
\newtheorem{problem}{Problem}
\newtheorem{proposition}{Proposition}[section]
\newtheorem{lemma}{Lemma}[section]
\newtheorem{corollary}[theorem]{Corollary}
\newtheorem{example}{Example}[section]
\newtheorem{definition}[problem]{Definition}
\newcommand{\BEQA}{\begin{eqnarray}}
\newcommand{\EEQA}{\end{eqnarray}}
\newcommand{\define}{\stackrel{\triangle}{=}}
\theoremstyle{remark}
\newtheorem{rem}{Remark}
\begin{document}
\bibliographystyle{IEEEtran}
\vspace{3cm}
\title{\textbf{NCERT 12/10/6}}
\author{EE23BTECH11040-MANOJ KUMAR AMBATIPUDI$^{*}$% <-this % stops a space
}
\maketitle
\newpage
\bigskip
\renewcommand{\thefigure}{\theenumi}
\renewcommand{\thetable}{\theenumi}
\textbf{Question:}
\\
A beam of light consisting of two wavelengths, 650 nm and 520 nm, is used to
obtain interference fringes in a Young\text{'}s double-slit experiment.\\
(a) Find the distance of the third bright fringe on the screen from
the central maximum for wavelength 650 nm.\\
(b) What is the least distance from the central maximum where the
bright fringes due to both the wavelengths coincide?
\\
\textbf{Solution:}
\begin{table}[h!]
\renewcommand\thetable{1}
    \centering
    \begin{tabular}{|c|c|c|}
    \hline
       \textbf{Variable}& \textbf{Description}& \textbf{Value}\\\hline
         $y_1$& Wave Equation of First Wave&none\\\hline
          $y_2$&Wave Equation of Second Wave &none\\\hline
         $Y$&   Resultant Wave&none\\\hline
          $f_c$& Frequency of wave&none\\\hline
         $A$& Amplitude of Wave&none\\\hline
         $\lambda_{1}$,$\lambda_2$& Wavelengths&650,520\\\hline
         $d$& distance between the 2 slits&None\\\hline
         $D$& Distance from slits and Screen&None\\\hline
    \end{tabular}
    \caption{\textbf{VARIABLES AND THEIR VALUES}}
    \label{tab_12/10/6/1}
\end{table}

Consider single source interference. Let the wave equations of the 2 waves coming from $\lambda$ source be $y_1$,$y_2$.
Assume an impulse emitted at source, travelling through the path labeled \textbf{Wave 1} be $t_1$ and for travelling through path \textbf{Wave 2} be $t_2$ respectively. The corresponding wave equations are
\begin{align}
    y_1 = Asin(2\pi f_c t_1)\\
    y_2 = Asin(2\pi f_c t_2)\label{12/10/6/1}
\end{align}
$t_2$ can be written in terms of $t_1$ as
\begin{align}
    t_2 = t_1 + \Delta t
\end{align}
Where the term $\Delta t$ arises because of paths chosen.
the values of $t_1, t_2$ are 
\begin{align}
    t_1 &= \frac{\sqrt{\brak{\dfrac{d}{2}-b_1}^{2}+(a_1)^{2}} + \sqrt{\brak{\dfrac{d}{2}-y}^{2}+(D)^{2}}}{c}\\
    t_2 &= \frac{\sqrt{\brak{\dfrac{d}{2}+y}^{2}+(D)^{2}} + \sqrt{\brak{\dfrac{d}{2}+b_1}^{2}+(a_1)^{2}}}{c}
\end{align}
Assuming the approximations 
\begin{align}
    b_1&<<a_1\\
    b_1&\sim d\\
    d&<<D\\
    \brak{1+x}^{n}&\simeq1+nx
\end{align}
We get
\begin{align}
    \Delta t &= \frac{\frac{db_1}{a_1} + \frac{dy}{D}}{c} \label{12/10/6/2}
\end{align}
On substituting $t_2$ in \eqref{12/10/6/1} and applying superposition theorem, we get
\begin{align}
    Y &= y_1 + y_2\\
    Y &= 2A\sin\brak{2\pi f_c\brak{t_1 + \frac{\Delta t}{2}}}\cos\brak{2\pi f_c\brak{\frac{\Delta t}{2}}}
\end{align}
For constructive interference
\begin{align}
     f \Delta t = n
\end{align}
substituting in \eqref{12/10/6/2}, we get
\begin{align}
    \frac{db_1}{a_1} + \frac{dy}{D} = n\lambda \label{12/10/6/3}
\end{align}
The equation \eqref{12/10/6/3} is the condition for constructive interference.
For 2 sources, we do it as follows.
From \figref{fig_12/10/6/2}, 
\begin{align}
    \frac{db_1}{a_1} + \frac{dy}{D} = n_1\lambda_1\\
    \frac{db_2}{a_2} + \frac{dy}{D} = n_2\lambda_2
\end{align}
Now, for lights from both sources to interfere at a point constructively, 
\begin{align}
    n_1\lambda_1 = n_2\lambda_2 \label{12/10/6/4}
\end{align}
(a) From \tabref{tab_12/10/6/1} and \eqref{12/10/6/3}, taking $b_1 = 0$
\begin{align}
    n_1 &= 3\\
    y &= \frac{n_1\lambda_1 D}{d}\\
    y &= 1950\frac{D}{d}
\end{align}
(b) From \tabref{tab_12/10/6/1}, \eqref{12/10/6/3}, \eqref{12/10/6/4} and taking $b_1 = 0$, $b_2 = 0$
\begin{align}
  n_1 \lambda_1 &= n_2\lambda_2 \\
\frac{n_2}{n_1} &= \frac{\lambda_1}{\lambda_2} = \frac{650}{520}\\
\frac{n_2}{n_1} &= \frac{5}{4}
\end{align}
Now we can assume $n_2$ = 5k and $n_1$ = 4k where k is some positive integer.
For smallest value of $y$ we shall take $k=1$.
\begin{align}
    y =\Delta x_1 \dfrac{D}{d} = 2600\dfrac{D}{d}nm 
\end{align}
\begin{figure}[h!]
\renewcommand\thefigure{1}
    \centering
    \begin{tikzpicture}
  % Screen
  \draw (-3,0) -- (6,0);
  % Double Slits
  \draw (0,3) -- (0,-3);
  \draw (4,4) -- (4,-4);
  % Light source
  \draw (-1,1) -- (0,2);
  \draw (-1,1) -- (0,-2);
  % Labels
  \node at (-1,1.3){$\lambda$,$(a_1,b_1)$};
  \node at (0,2.2){$(0,\frac{d}{2})$};
  \node at (0,-2.2){$(0,\frac{-d}{2})$};
  \node at (4.5,2.2){(D,y)};
  % Dashed lines for waves
  \draw[dashed] (0,2) -- (4,2);
  \draw[dashed] (0,-2) -- (4,2);
  % Labels for waves
  \node at (2.5,2.5) {Wave 1};
  \node at (2.5,-1) {Wave 2};
\end{tikzpicture}
    \caption{Single Source YDSE}
    \label{fig_12/10/6/2}
\end{figure}



\begin{figure}[h!]
\renewcommand\thefigure{2}
    \centering
   \begin{tikzpicture}
  % Screen
  \draw (-3,0) -- (6,0);
  % Double Slits
  \draw (0,3) -- (0,-3);
  \draw (4,4) -- (4,-4);
  % Light source
  \draw (-1,1) -- (0,2);
  \draw (-1,1) -- (0,-2);
  \draw (-2,1.5) -- (0,2);
  \draw (-2,1.5) -- (0,-2);
  % Labels
  \node at (0.5,0.3) {Sources};
  \node at (-1,1.3){$\lambda_1$,$(a_1,b_1)$};
  \node at (-2,1.8){$\lambda_2$,$(a_2,b_2)$};
  \node at (0,2.2){$(0,\dfrac{d}{2})$};
  \node at (0,-2.2){$(0,\dfrac{-d}{2})$};
  \node at (4,2.2){(D,0)};
  % Dashed lines for waves
  \draw[dashed] (0,2) -- (4,2);
  \draw[dashed] (0,-2) -- (4,2);
  % Labels for waves
  \node at (2.5,2.5) {Wave 1};
  \node at (2.5,-1) {Wave 2};
\end{tikzpicture}
    \caption{2 sources interference}
    \label{fig_12/10/6/2}
\end{figure}

\end{document}
