\let\negmedspace\undefined
\let\negthickspace\undefined
\documentclass[journal,12pt,twocolumn]{IEEEtran}
\usepackage{cite}
\usepackage{amsmath,amssymb,amsfonts,amsthm}
\usepackage{algorithmic}
\usepackage{graphicx}
\usepackage{textcomp}
\usepackage{xcolor}
\usepackage{txfonts}
\usepackage{listings}
\usepackage{enumitem}
\usepackage{mathtools}
\usepackage{gensymb}
\usepackage{comment}
\usepackage[breaklinks=true]{hyperref}
\usepackage{tkz-euclide} 
\usepackage{listings}
\usepackage{gvv}                                        
\def\inputGnumericTable{}                                 
\usepackage[latin1]{inputenc}                                
\usepackage{color}                                            
\usepackage{array}                                            
\usepackage{longtable}                                       
\usepackage{calc}                                             
\usepackage{multirow}                                         
\usepackage{hhline}                                           
\usepackage{ifthen}                                           
\usepackage{lscape}

\newtheorem{theorem}{Theorem}[section]
\newtheorem{problem}{Problem}
\newtheorem{proposition}{Proposition}[section]
\newtheorem{lemma}{Lemma}[section]
\newtheorem{corollary}[theorem]{Corollary}
\newtheorem{example}{Example}[section]
\newtheorem{definition}[problem]{Definition}
\newcommand{\BEQA}{\begin{eqnarray}}
\newcommand{\EEQA}{\end{eqnarray}}
\newcommand{\define}{\stackrel{\triangle}{=}}
\theoremstyle{remark}
\newtheorem{rem}{Remark}

\begin{document}

\bibliographystyle{IEEEtran}
\vspace{3cm}

\title{\textbf{12.10.6}}
\author{EE23BTECH11040-MANOJ KUMAR AMBATIPUDI$^{*}$% <-this % stops a space
}
\maketitle
\newpage
\bigskip

\renewcommand{\thefigure}{\theenumi}
\renewcommand{\thetable}{\theenumi}

\textbf{Question:}
\\
A beam of light consisting of two wavelengths, 650 nm and 520 nm, is used to
obtain interference fringes in a Young\text{'}s double-slit experiment.\\
(a) Find the distance of the third bright fringe on the screen from
the central maximum for wavelength 650 nm.\\
(b) What is the least distance from the central maximum where the
bright fringes due to both the wavelengths coincide?
\\

\textbf{Solution:}
\\
(a) Finding the distance of the third bright fringe on the screen from the central maximum for wavelength
\begin{align}
\lambda_1 = 650 \, \text{nm}
\end{align}
The path difference for constructive interference is given by:
\begin{align}
\Delta x_1 = m\lambda_1     
\end{align}
where  m = 3  for the third bright fringe.

Substitute the values:
\begin{align}
\Delta x_1 &= 3\times650 = 1950 \, \text{nm} \\
y &= \Delta x_1\cdot\dfrac{D}{d} = 1950\times\dfrac{D}{d}
\end{align}

Where "D" is the distance of screen from sources, "d" is the distance between sources and "y" is the distance from Central Maxima.\\


(b) The least distance from the central maximum where the bright fringes due to both wavelengths coincide:

Given \[\lambda_1 = 650 \text{nm} ,\lambda_1 = 520 \text{nm}\]

Let $\Delta x$  be the common path difference for both wavelengths:
\begin{align}
\Delta x &= m \lambda_1 = n\lambda_2 \\
\frac{n}{m} &= \frac{\lambda_1}{\lambda_2} = \frac{650}{520}\\
\frac{n}{m} &= \frac{5}{4}
\end{align}
Now we can assume n = 5 and m = 4.\\
Path difference $\Delta x = 4\times650 = 2600 \text{nm}$\\\\
Now $y = \Delta x\cdot \dfrac{D}{d} = 2600\times \dfrac{D}{d}$\\\\
where "y" is the distance from central maxima, "D" is the distance of screen from sources and "d" is the distance between sources.

\textbf{Answer for (b):}
\\
Hence, at a least distance of $2600\times\dfrac{D}{d}$ the bright fringes due to  both wavelengths coincide.
\end{document}

