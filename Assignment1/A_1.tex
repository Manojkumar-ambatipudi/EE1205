\let\negmedspace\undefined
\let\negthickspace\undefined
\documentclass[journal,12pt,twocolumn]{IEEEtran}
\usepackage{cite}
\usepackage{amsmath,amssymb,amsfonts,amsthm}
\usepackage{algorithmic}
\usepackage{graphicx}
\usepackage{textcomp}
\usepackage{xcolor}
\usepackage{txfonts}
\usepackage{listings}
\usepackage{enumitem}
\usepackage{mathtools}
\usepackage{gensymb}
\usepackage{comment}
\usepackage[breaklinks=true]{hyperref}
\usepackage{tkz-euclide} 
\usepackage{listings}
\usepackage{gvv}                                        
\def\inputGnumericTable{}                                 
\usepackage[latin1]{inputenc}                                
\usepackage{color}                                            
\usepackage{array}                                            
\usepackage{longtable}                                       
\usepackage{calc}                                             
\usepackage{multirow}                                         
\usepackage{hhline}                                           
\usepackage{ifthen}                                           
\usepackage{lscape}
\newtheorem{theorem}{Theorem}[section]
\newtheorem{problem}{Problem}
\newtheorem{proposition}{Proposition}[section]
\newtheorem{lemma}{Lemma}[section]
\newtheorem{corollary}[theorem]{Corollary}
\newtheorem{example}{Example}[section]
\newtheorem{definition}[problem]{Definition}
\newcommand{\BEQA}{\begin{eqnarray}}
\newcommand{\EEQA}{\end{eqnarray}}
\newcommand{\define}{\stackrel{\triangle}{=}}
\theoremstyle{remark}
\newtheorem{rem}{Remark}
\begin{document}
\bibliographystyle{IEEEtran}
\vspace{3cm}
\title{\textbf{12.10.6}}
\author{EE23BTECH11040-MANOJ KUMAR AMBATIPUDI$^{*}$% <-this % stops a space
}
\maketitle
\newpage
\bigskip
\renewcommand{\thefigure}{\theenumi}
\renewcommand{\thetable}{\theenumi}
\textbf{Question:}
\\
A beam of light consisting of two wavelengths, 650 nm and 520 nm, is used to
obtain interference fringes in a Young\text{'}s double-slit experiment.\\
(a) Find the distance of the third bright fringe on the screen from
the central maximum for wavelength 650 nm.\\
(b) What is the least distance from the central maximum where the
bright fringes due to both the wavelengths coincide?
\\
\textbf{Solution:}
\\
Let the wave equations of the 2 waves coming to interfere be \begin{align}
    y_1 &= A_1sin(2\pi f_1t) + A_2sin(s\pi f_2t)\\
    y_2 &= A_1sin(2\pi f_1t +\phi_1)+A_2sin(2\pi f_2t + \phi_2)
\end{align}
Where $A_1$ and $A_2$ are amplitudes of waves and $f_c$ is the frequency of the waves. 
Using principle of superposition, we get 
\begin{align}
    y &= y_1+y_2 
\end{align}
where $y$ is the resultant wave and $y_1$,$y_2$ are initial waves.
\begin{align}
   y&=A_1(sin(2\pi f_1t)+sin(2\pi f_1t+\phi_1))+\\
   &A_2(sin(2\pi f_2t)+sin(2\pi f_2t+\phi_2))
\end{align}
using
\begin{align}
    sin(c)+sin(d)=2sin(\dfrac{c+d}{2})cos(\dfrac{c-d}{2})
\end{align}
we get
\begin{align}
    y&=2A_1sin(2\pi f_1t + \dfrac{\phi_1}{2})cos(\dfrac{\phi_1}{2})+\\&2A_2sin(2\pi f_2t + \dfrac{\phi_2}{2})cos(\dfrac{\phi_2}{2})
\end{align}
Now for constructive interference to happen,
\begin{align*}
    cos(\dfrac{\phi_1}{2})=+/-1 \text{ and } cos(\dfrac{\phi_2}{2})=+/-1 
\end{align*}
\begin{align}
\implies \phi_1=2n_1\pi \text{ and } \phi_2=2n_2\pi
\end{align}
Equation (9) is the condition for constructive Interference.\\
Now we will derive the condition for constructive interference in a YDSE setup where the light waves due to both the sources coincide.\\
The path difference suffered by 2 waves due to corresponding phase difference is given by,
\begin{align}
    \Delta x_1&=\dfrac{\lambda_1}{2\pi}2n_1\pi=n_1\lambda\\
    \Delta x_2&=\dfrac{\lambda_2}{2\pi}2n_2\pi=n_2\lambda
\end{align}
In a YDSE setup, the 2 sources are seperated by a distance "d", the distance between screen and mid point of sources is "D" and $\theta_1$ is the angle made by point of interest with horizontal line. The path difference between the 2 waves interfereing at the point of interest is given by 
\begin{align}
    \Delta x_1 = dsin(\theta_1)\\
    \Delta x_2 = dsin(\theta_2)
\end{align}
Now, from (12) and (13), we can write,
\begin{align}
    dsin(\theta_1)=n\lambda_1\\
    dsin(\theta_2)=n\lambda_2
\end{align}
Now, for small values of $\theta_1$ and $\theta_2$, we can approximate 
\begin{align}
    sin(\theta_1)=\dfrac{y_1}{D}\\
    sin(\theta_2)=\dfrac{y_2}{D}
\end{align}
upon substituting in (14),(15) and rearranging, we get
\begin{align}
    y_1=n_1\dfrac{D\lambda_1}{d}\\
    y_2=n_2\dfrac{D\lambda_2}{d}
\end{align}
Now for interference to happen at same points,
\begin{align}
         y_1&=y_2\\
\implies n_1\lambda_1&=n_2\lambda_2
\end{align}
Now,we shall use the above equations to solve the questions.
\\
(a) Finding the distance of the third bright fringe on the screen from the central maximum for wavelength
\begin{align}
\lambda_1 = 650 \, \text{nm}
\end{align}
The path difference for constructive interference is given by:
\begin{align}
\Delta x_1 = n_1\lambda_1     
\end{align}
where  $n_1$ = 3  for the third bright fringe.

Substitute the values:
\begin{align}
\Delta x_1 &= 3\times650 = 1950 \, \text{nm} \\
y_1 &= \Delta x_1\cdot\dfrac{D}{d} = 1950\times\dfrac{D}{d}
\end{align}

Where "D" is the distance of screen from sources, "d" is the distance between sources and "$y_1$" is the distance from Central Maxima.\\
(b) The least distance from the central maximum where the bright fringes due to both wavelengths coincide:
Given \[\lambda_1 = 650 \text{nm} ,\lambda_1 = 520 \text{nm}\]
\begin{align}
  n_1 \lambda_1 &= n_2\lambda_2 \\
\frac{n_2}{n_1} &= \frac{\lambda_1}{\lambda_2} = \frac{650}{520}\\
\frac{n_2}{n_1} &= \frac{5}{4}
\end{align}
Now we can assume $n_2$ = 5k and $n_1$ = 4k where k is some positive integer.\\
For smallest value of $y_1$ we shall take k=1.\\
Path difference $\Delta x_1 = 4\times650 = 2600 \text{nm}$\\\\
Now $y_1 =\Delta x_1 \dfrac{D}{d} = 2600\dfrac{D}{d}nm $\\
\textbf{Answer for (b):}
\\
Hence, at a least distance of $2600\times\dfrac{D}{d}$nm the bright fringes due to  both wavelengths coincide.
\begin{table}
    \centering
    \begin{tabular}{|c|c|c|c|}
    \hline
       \textbf{Variable}& \textbf{Description}& \textbf{Value}\\\hline
         $y_1$& Wave Equation of First Wave&none\\\hline
          $y_2$&Wave Equation of Second Wave &none\\\hline
         $y$&   Resultant Wave&none\\\hline
          $f_c$& Frequency of waves&none\\\hline
         $A_1$& Amplitude of First Wave&none\\\hline
         $A_2$& Amplitude of Second Wave&none\\\hline
         $\phi_1$,$\phi_2$& Phase Difference(before Interference)&none\\\hline
         $I_{net}$& Intensity after interference&none\\\hline
         $n_1,n_2$& Non Negative Integer Values&0,1,2....\\\hline
         $\Delta x_1$,$\Delta x_2$& Path Differences&In soln.\\\hline
         $\lambda_{1}$,$\lambda_2$& Wavelengths&650,520\\\hline
         $y_1,y_2$& Distance between central maxima and point&In soln.\\\hline
         $d$& distance between the slits&None\\\hline
         $D$& Distance between Slits and Screen&None\\\hline
         $\theta_1,\theta_2$& Angular Distance From Central Maxima&None\\\hline
    \end{tabular}
    \caption{\textbf{VARIABLES AND THEIR VALUES}}
    \label{tab:my_label}
\end{table}
\end{document}
