\let\negmedspace\undefined
\let\negthickspace\undefined
\documentclass[journal,12pt,twocolumn]{IEEEtran}
\usepackage{cite}
\usepackage{amsmath,amssymb,amsfonts,amsthm}
\usepackage{algorithmic}
\usepackage{graphicx}
\usepackage{textcomp}
\usepackage{xcolor}
\usepackage{txfonts}
\usepackage{listings}
\usepackage{enumitem}
\usepackage{mathtools}
\usepackage{gensymb}
\usepackage{comment}
\usepackage[breaklinks=true]{hyperref}
\usepackage{tkz-euclide} 
\usepackage{listings}
\usepackage{gvv}                                        
\def\inputGnumericTable{}                                 
\usepackage[latin1]{inputenc}                                
\usepackage{color}                                            
\usepackage{array}                                            
\usepackage{longtable}                                       
\usepackage{calc}                                             
\usepackage{multirow}                                         
\usepackage{hhline}                                           
\usepackage{ifthen}                                           
\usepackage{lscape}

\newtheorem{theorem}{Theorem}[section]
\newtheorem{problem}{Problem}
\newtheorem{proposition}{Proposition}[section]
\newtheorem{lemma}{Lemma}[section]
\newtheorem{corollary}[theorem]{Corollary}
\newtheorem{example}{Example}[section]
\newtheorem{definition}[problem]{Definition}
\newcommand{\BEQA}{\begin{eqnarray}}
\newcommand{\EEQA}{\end{eqnarray}}
\newcommand{\define}{\stackrel{\triangle}{=}}
\theoremstyle{remark}
\newtheorem{rem}{Remark}

\begin{document}

\bibliographystyle{IEEEtran}
\vspace{3cm}

\title{\textbf{12.10.6}}
\author{EE23BTECH11040-MANOJ KUMAR AMBATIPUDI$^{*}$% <-this % stops a space
}
\maketitle
\newpage
\bigskip

\renewcommand{\thefigure}{\theenumi}
\renewcommand{\thetable}{\theenumi}

\textbf{Question:}
\\
A beam of light consisting of two wavelengths, 650 nm and 520 nm, is used to
obtain interference fringes in a Young\text{'}s double-slit experiment.\\
(a) Find the distance of the third bright fringe on the screen from
the central maximum for wavelength 650 nm.\\
(b) What is the least distance from the central maximum where the
bright fringes due to both the wavelengths coincide?
\\

\textbf{Solution:}
\\
Let the wave equations of the 2 waves coming to interfere be \begin{align}
    y_1 = A_1sin(2\pi f_ct)\\
    y_2 = A_2sin(2\pi f_ct +\phi)
\end{align}
Where $A_1$ and $A_2$ are amplitudes of waves and $f_c$ is the frequency of the waves. 
Using principle of superposition, we get 
\begin{align}
    y &= y_1+y_2 
\end{align}
where $y$ is the resultant wave and $y_1$,$y_2$ are initial waves. Expanding $sin(2\pi f_ct+\phi)$, we get 
\begin{align}
    y&=(A_1+A_2cos(\phi))sin(2\pi f_ct)+(A_2sin(\phi))sin(2\pi f_ct)
\end{align}
Assuming
\begin{align}
    A_1+A_2cos(\phi)&=Rcos(\theta)\\
    A_2sin(\phi)&=Rsin(\theta)
\end{align}
here R is the resultant and $\theta$ is the phase difference
we get,
\begin{align}
    y&=Rsin(2\pi f_ct+\theta)
\end{align}
and 
\begin{align}
    R^2&=A_1^2+A_2^2+2A_1A_2cos(\theta)
\end{align}
For any Electromagnetic Wave, \\
Intensity(I) $\alpha$ $(\text{Amplitude})^2$
\begin{align}
    I_{net} &= I_1+I_2+2\sqrt{I_1I_2}cos(\theta)
\end{align}
From (9), it is clear that maximum intensity occurs when
\begin{align}
    \theta = 2n\pi
\end{align}
For any wave, phase difference $\theta$ is given by
\begin{align}
    \theta = \frac{2\pi\Delta x}{\lambda}
\end{align}
Now, path difference $\Delta x$ is given as
\begin{align}
    \Delta x &=\frac{\lambda}{2\pi}\theta\\
    \implies\Delta x &= \frac{\lambda}{2\pi}2n\pi\\
    \implies\Delta x &= n\lambda
\end{align}
Equation (14) is the condition for constructive Interference.\\
Now we will derive the condition for constructive interference in a YDSE setup.\\

In a YDSE setup, the 2 sources are seperated by a distance "d", the distance between screen and mid point of sources is "D" and $\theta_1$ is the angle made by point of interest with horizontal line. The path difference between the 2 waves interfereing at the point of interest is given by 
\begin{align}
    \Delta x = dsin(\theta_1)
\end{align}
Now, from (14) and (15), we can write,
\begin{align}
    dsin(\theta_1)=n\lambda
\end{align}
Now, for small values of $\theta_1$, we can approximate 
\begin{align}
    sin(\theta_1)=\dfrac{y}{D}
\end{align}
upon substituting in (16) and rearranging, we get
\begin{align}
    y=n\dfrac{D\lambda}{d}
\end{align}

Now,we shall use the above equation to solve the questions.
\\
(a) Finding the distance of the third bright fringe on the screen from the central maximum for wavelength
\begin{align}
\lambda_1 = 650 \, \text{nm}
\end{align}
The path difference for constructive interference is given by:
\begin{align}
\Delta x_1 = m\lambda_1     
\end{align}
where  m = 3  for the third bright fringe.

Substitute the values:
\begin{align}
\Delta x_1 &= 3\times650 = 1950 \, \text{nm} \\
y &= \Delta x_1\cdot\dfrac{D}{d} = 1950\times\dfrac{D}{d}
\end{align}

Where "D" is the distance of screen from sources, "d" is the distance between sources and "y" is the distance from Central Maxima.\\


(b) The least distance from the central maximum where the bright fringes due to both wavelengths coincide:

Given \[\lambda_1 = 650 \text{nm} ,\lambda_1 = 520 \text{nm}\]

Let $\Delta x$  be the common path difference for both wavelengths:
\begin{align}
\Delta x &= n_1 \lambda_1 = n_2\lambda_2 \\
\frac{n_2}{n_1} &= \frac{\lambda_1}{\lambda_2} = \frac{650}{520}\\
\frac{n_2}{n_1} &= \frac{5}{4}
\end{align}
Now we can assume $n_2$ = 5 and $n_1$ = 4.\\
Path difference $\Delta x = 4\times650 = 2600 \text{nm}$\\\\
Now $y = \Delta x\cdot \dfrac{D}{d} = 2600\times \dfrac{D}{d}$\\\\
where "y" is the distance from central maxima, "D" is the distance of screen from sources and "d" is the distance between sources.

\textbf{Answer for (b):}
\\
Hence, at a least distance of $2600\times\dfrac{D}{d}$ the bright fringes due to  both wavelengths coincide.

\begin{table}
    \centering
    \begin{tabular}{|c|c|c|c|}
    \hline
       \textbf{Variable}& \textbf{Description}& \textbf{Value}\\\hline
         $y_1$& Wave Equation of First Wave&none\\\hline
          $y_2$&Wave Equation of Second Wave &none\\\hline
         $y$&   Resultant Wave&none\\\hline
          $f_c$& Frequency of waves&none\\\hline
         $A_1$& Amplitude of First Wave&none\\\hline
         $A_2$& Amplitude of Second Wave&none\\\hline
         $\phi$& Phase Difference(before Interference)&none\\\hline
         $\theta$& Phase Difference (After Interference)&none\\\hline
         R& Amplitude After Interference&none\\\hline
         $I_{net}$& Intensity after interference&none\\\hline
         $I_1$& Intensity of First Wave&none\\\hline
         $I_2$& Intensity of First Wave&none\\\hline
         $n,n_1,n_2$& Non Negative Integer Values&0,1,2....\\\hline
         $\Delta x$& Path Difference&In soln.\\\hline
         $\lambda$& Wavelength&none\\\hline
         $y$& Distance between central maxima and point&In soln.\\\hline
         $d$& distance between the slits&None\\\hline
         $D$& Distance between Slits and Screen&None\\\hline
         $\theta_1$& Angular Distance From Central Maxima&None\\\hline
        $\lambda_1$& Wavelength of First Line&650nm\\\hline
        $\lambda_2$& Wavelength of second line&520nm\\\hline
    \end{tabular}
    \caption{\textbf{VARIABLES AND THEIR VALUES}}
    \label{tab:my_label}
\end{table}

\end{document}

